\section{geneparameter}
\label{module_geneparameter}
This module contains the following objects

\begin{itemize}
\item \ref{object_GeneParameter21} GeneParameter21

\end{itemize}
\subsection{Object GeneParameter21}

\label{object_GeneParameter21}

The GeneParameter21 object has the following fields. To see how to access them refer to \ref{accessing_fields}
\begin{description}
\item{gp} Type [GeneParser21 * : Scalar] No documentation

\item{cm} Type [CodonMapper  * : Scalar] No documentation

\item{cses} Type [ComplexSequenceEvalSet * : Scalar] No documentation

\item{ss} Type [SpliceSiteModel ** : List]  held only to be free'd when GeneParser21Set is free'd

\item{rc} Type [RandomCodon  * : Scalar]  needed to soak up the odd-and-sods of genes

\item{gwcm} Type [GeneWiseCodonModel * : Scalar] No documentation

\item{ct} Type [CodonTable   * : Scalar] No documentation

\item{modelled_splice} Type [boolean : Scalar]  so we can alter balance scores.

\item{gms} Type [GeneModel    * : Scalar] No documentation

\end{description}


GeneParameter21 keeps all the parameters
for genewise algorithms in one tidy unit.


This is also the switch between the old (compugen handled)
and new statistics. This object can be made from
either the old or the new statistics


I have made the object complete opaque to 
scripting apis because the contents have to
be coordinated quite well




Member functions of GeneParameter21

