\section{threestatedb}
\label{module_threestatedb}
This module contains the following objects

\begin{itemize}
\item \ref{object_ThreeStateDB} ThreeStateDB

\end{itemize}
\subsection{Object ThreeStateDB}

\label{object_ThreeStateDB}

The ThreeStateDB object has the following fields. To see how to access them refer to \ref{accessing_fields}
\begin{description}
\item{dbtype} Type [int : Scalar] No documentation

\item{filename} Type [char * : Scalar] No documentation

\item{type} Type [int : Scalar] No documentation

\item{current_file} Type [FILE * : Scalar] No documentation

\item{rm} Type [RandomModel * : Scalar]  NB, this is hard-linked... 

\item{byte_position} Type [long : Scalar]  this is the file position for the current model

\item{single} Type [ThreeStateModel * : Scalar]  for single db cases

\item{phdb} Type [PfamHmmer1DB * : Scalar]  for Pfam Hmmer1 style databases.

\item{sdb} Type [SequenceDB   * : Scalar]  for protein databases

\item{comp} Type [CompMat      * : Scalar]  for protein databases

\item{gap} Type [int : Scalar]  for protein databases

\item{ext} Type [int : Scalar]  for protein databases

\item{seq_cache} Type [Sequence *  : Scalar]  needed for a bit of inter-function communication 

\item{reload_generic} Type [ThreeStateModel * (*reload_generic)(ThreeStateDB * tdb,int * return_status) : Scalar]  for generic parsing applications...


\item{open_generic} Type [boolean (*open_generic)(ThreeStateDB * tdb) : Scalar] No documentation

\item{close_generic} Type [boolean (*close_generic)(ThreeStateDB * tdb) : Scalar] No documentation

\item{dataentry_add} Type [boolean (*dataentry_add)(ThreeStateDB * tdb,DataEntry * en) : Scalar] No documentation

\item{open_index_generic} Type [boolean (*open_index_generic)(ThreeStateDB *tdb) : Scalar] No documentation

\item{index_generic} Type [ThreeStateModel * (*index_generic)(ThreeStateDB *tdb,DataEntry *de) : Scalar] No documentation

\item{close_index_generic} Type [boolean (*close_index_generic)(ThreeStateDB *tdb) : Scalar] No documentation

\item{data} Type [void * : Scalar]  whatever else the damn system wants to carry around with it! 

\end{description}
ThreeStateDB is the object that represents
a database of profile-HMMs. 


The object hold a variety of fields on some of which are
occupied depending on the type.


Realistically we need a more abstract class idea, which is
implemented here anyway via the generic stuff, in hacky
C-style pointers to function plus a void pointer. This object
therefore houses a switch system around the different types
including the generic system... but as the generic function
stuff was bolted on later, some things are handled with
explicit datastructures. It is quite messy ;). Apologies.
To be cleaned up.


The generic stuff was principly added in to allow a decoupling of this module
from the HMMer2.o interface code which is held in wise2xhmmer.dy


The old static datastructure code can be 
made via protein sequences which are then converted or a 
Pfam 2.0 style directory + HMMs file.




Member functions of ThreeStateDB

\subsubsection{indexed_ThreeStateModel_ThreeStateDB}

\begin{description}
\item[External C] {\tt Wise2_indexed_ThreeStateModel_ThreeStateDB (mdb,en)}
\item[Perl] {\tt &Wise2::ThreeStateDB::indexed_model (mdb,en)}

\item[Perl-OOP call] {\tt $obj->indexed_model(en)}

\end{description}
Arguments
\begin{description}
\item[mdb] [UNKN ] database where this is indexed [ThreeStateDB *]
\item[en] [UNKN ] dataentry to pull the model from [DataEntry *]
\item[returns] [UNKN ] Undocumented return value [ThreeStateModel *]
\end{description}
Retrieves a model from a database which has been opened
for indexing by /open_for_indexing_ThreeStateDB


The index information comes from the dataentry which should 
have been from a search of the ThreeStateDB.


\subsubsection{new_proteindb_ThreeStateDB}

\begin{description}
\item[External C] {\tt Wise2_new_proteindb_ThreeStateDB (sdb,comp,gap,ext)}
\item[Perl] {\tt &Wise2::ThreeStateDB::new_proteindb_ThreeStateDB (sdb,comp,gap,ext)}

\item[Perl-OOP call] {\tt $obj->new_proteindb_ThreeStateDB(comp,gap,ext)}

\end{description}
Arguments
\begin{description}
\item[sdb] [READ ] sequence database to use [SequenceDB *]
\item[comp] [READ ] comparison matrix to use [CompMat *]
\item[gap] [READ ] gap open penalty [int]
\item[ext] [READ ] gap extensions penalty [int]
\item[returns] [UNKN ] Undocumented return value [ThreeStateDB *]
\end{description}
makes a new ThreeStateDB from a
sequencedb (better be protein!)




\subsubsection{new_PfamHmmer1DB_ThreeStateDB}

\begin{description}
\item[External C] {\tt Wise2_new_PfamHmmer1DB_ThreeStateDB (dirname)}
\item[Perl] {\tt &Wise2::ThreeStateDB::new_PfamHmmer1DB_ThreeStateDB (dirname)}

\item[Perl-OOP call] {\tt $obj->new_PfamHmmer1DB_ThreeStateDB()}

\end{description}
Arguments
\begin{description}
\item[dirname] [UNKN ] Undocumented argument [char *]
\item[returns] [UNKN ] Undocumented return value [ThreeStateDB *]
\end{description}
Makes a new PfamHmmer1DB from a filename
indicating the directory


\subsubsection{new_single_ThreeStateDB}

\begin{description}
\item[External C] {\tt Wise2_new_single_ThreeStateDB (tsm,rm)}
\item[Perl] {\tt &Wise2::ThreeStateDB::new_single_ThreeStateDB (tsm,rm)}

\item[Perl-OOP call] {\tt $obj->new_single_ThreeStateDB(rm)}

\end{description}
Arguments
\begin{description}
\item[tsm] [READ ] a single ThreeStateModel [ThreeStateModel *]
\item[rm] [READ ] random model to be used in comparisons.. [RandomModel *]
\item[returns] [UNKN ] Undocumented return value [ThreeStateDB *]
\end{description}
Making a new ThreeStateDB from a single
model




